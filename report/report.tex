\documentclass{scrartcl}

\usepackage{amsfonts}
\usepackage{multicol}
\usepackage{hyperref}
\hypersetup{
    colorlinks=true,
    linkcolor=blue,
    filecolor=magenta,      
    urlcolor=cyan,
    pdfpagemode=FullScreen,
}

\subject{GPU Computing}
\title{First Assignment}
\subtitle{Report}
\author{Christian Dalvit}

\begin{document}
    \maketitle
    \section{Introduction}    
    The goal of this homework is to implement an algorithm that transposes a non-symmetric matrix. Furthermore, different metrics of the algorithm should be measured and analyzed. In this report I describe the problem setting, algorithms and experimental results of my implementation.\\
    The code used for this homework is made available through a public \href{https://github.com/chrisdalvit/matrix-transpose-benchmark}{Github repository}. Details on how to run the code and reproduce the results can be found in the \texttt{README.md} file of the Github repository.

    \section{Problem Description}
    For a given matrix $A \in \mathbb{R}^{n \times m}$, the transpose of the matrix $A^T \in \mathbb{R}^{m \times n}$ is defined as
    $$
        A^T_{ij} = A_{ji}
    $$
    In this homework, matrices have dimensions of $2^N$ for $N \in \mathbb{N}$, so only square matrices are considered. As a result, the implemented algorithms don't need to accommodate changes in the output matrix's shape.

    While implementing an algorithm that computes the transpose of a matrix is straightforward, comming up with an efficient implementation is quite tricky. Generally, the exploitation of spatial and temporal locality can enhance efficiency.

    \subsection{Algorithms}
    All human things are subject to decay. And when fate summons, Monarchs must obey.

    \section{Experiments}
    \subsection{Setup}
    \subsection{Results}
    All human things are subject to decay. And when fate summons, Monarchs must obey.
\end{document}